% ------------------------------------------------------------------------------
% TYPO3 Version 11.3 - What's New (German Version)
%
% @license	Creative Commons BY-NC-SA 3.0
% @link		https://typo3.org/help/documentation/whats-new/
% @language	German
% ------------------------------------------------------------------------------
% TYPO3 Version 11.3 - Executive Summary

\begin{frame}[fragile]
	\frametitle{Einführung}
	\framesubtitle{Zusammenfassung}

	\small
		TYPO3 Version 11.3 ist das vierte Sprint-Release auf dem Weg zur LTS-Version
		(long-term support) im Oktober 2021. Diese Version enthält eine Reihe von Backend
		UX-Verbesserungen und ist nun mit PHP Version 8.0 kompatibel. User experience
		(UX) befasst sich damit, wie Benutzer mit einem System oder einer Weboberfläche interagieren.

		\vspace{0.2cm}

		Neben der PHP-Version 7.4 ist der TYPO3 Core nun auch mit der PHP-Version 8.0
		kombatibel. Entwickler können viele neue Funktionen, Optimierungen und
		Verbesserungen der Programmiersprache nutzen, die TYPO3 zugrunde
		liegt.

		\vspace{0.1cm}

		Das nächste Release auf der Roadmap ist TYPO3 v11.4. Entwickler und Mitwirkende
		sollten sich bewusst sein, dass dieses TYPO3-Release (geplant für den 7. September 2021)
		den \textbf{feature freeze} für den v11-Zyklus markieren wird. Jetzt ist
		der beste Zeitpunkt, um Ihre Code-Beiträge an den TYPO3 Core zu übermitteln, wenn Sie
		diese in TYPO3 v11 LTS sehen wollen.
	\normalsize

\end{frame}

% ------------------------------------------------------------------------------
