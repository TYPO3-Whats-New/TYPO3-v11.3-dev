% ------------------------------------------------------------------------------
% TYPO3 Version 11.3 - What's New (English Version)
%
% @author	Michael Schams <schams.net>
% @license	Creative Commons BY-NC-SA 3.0
% @link		https://typo3.org/help/documentation/whats-new/
% @language	English
% ------------------------------------------------------------------------------
% ...

\begin{frame}[fragile]
	\frametitle{Extbase and Fluid}
	\framesubtitle{PSR-7 Request/Response}

	% decrease font size for code listing
	\lstset{basicstyle=\fontsize{8}{10}\ttfamily}

	\begin{itemize}
		\item The following Extbase class now implements the PSR-7 \texttt{ServerRequestInterface}:\newline
			\texttt{TYPO3\textbackslash
				CMS\textbackslash
				Extbase\textbackslash
				Mvc\textbackslash
				Request}

% The extbase TYPO3\CMS\Extbase\Mvc\Request now implements the PSR-7 ServerRequestInterface and
% thus holds all request related information of the main core request in addition to the plugin namespace specific extbase arguments.

% To further prepare extbase towards PSR-7 compatible requests, the extbase TYPO3\CMS\Extbase\Mvc\Request has to be streamlined.

% We have implemented unified standards and state-of-the-art technologies for developers in previous TYPO3 releases by following well-known PSR standards.
% We kept our promise to continuously improve the TYPO3 ecosystem and had a closer look at which areas we could improve further in this regard.

% As many TYPO3 developers know, Extbase is the object-oriented PHP framework heavily used by the TYPO3 Core and for extension development.
% As extension developers often access information about the current request in their custom code, a clean PSR-7 Request/Response handling is important.
% To let developers achieve this, Extbase now implements a “ServerRequestInterface”.
% Accessing request-related information from the core is as easy as pie.
% The details are all available through $this->request within Extbase controllers.
% This change also results in a streamlined class “\TYPO3\CMS\Extbase\Mvc\Request” and in several deprecations, for example the methods getRequestUri() and getBaseUri().

% \begin{lstlisting}
% <f:link.file file="{file}" download="true"
%   filename="alternative-name.jpg">Download</f:link.file>
% \end{lstlisting}

	\end{itemize}

\end{frame}

% ------------------------------------------------------------------------------
